% !TEX TS-program = XeLaTeX
% The next lines tell TeXShop to typeset with xelatex, and to open and save the source with Unicode encoding.

%!TEX encoding = UTF-8 Unicode
% !BIB TS-program =

\documentclass[12pt]{article}

\usepackage{uproposal}


\title{Coded Hashes of Arbitrary Images\\\normalsize \sf (or: the last frontier of emoji encoding)}
\author{
Steven R.~Loomis\\
\small srl@icu-project.org\\
\small (individual contribution)\\
\and
Keith Winstein\\
\small keithw@cs.stanford.edu\\
\small Stanford University\\
}

\date{2016-05-02\\\small\url{https://srl295.github.io}}                                           % Activate to display a given date or no date'




\makeindex

\begin{document}

\maketitle

\section{Introduction}
\addcontentsline{toc}{section}{Introduction}

\begin{quote}
Emoji are pictographs (pictorial symbols) that are typically presented
in a colorful cartoon form and used inline in text. [\ldots] In
Unicode 8.0, there is a total of 1,282 emoji, which are represented
using 1,051 code points.\autocite{UTR51}
\end{quote}

Recently, there has been considerable interest in adding newly created
pictorial symbols, not found in any existing character set, to the
Unicode Standard as emoji.\footnote{\textit{E.g.,}
  \href{https://www.change.org/p/unicode-consortium-the-taco-emoji-needs-to-happen-aeb4ebc7-a323-441d-90b9-20b90c83a8c6}{Taco
    Emoji campaign}, \href{http://www.beardemoji.com/}{Beard Emoji
    campaign}, \href{http://www.dumplingemoji.org}{Dumpling Emoji
    campaign}. There are currently 79 candidate emoji that have been
  assigned tentative code points.} Advocacy groups and others request
these code points because Unicode plain text remains the dominant
interoperable interchange format for messaging. In practice, before a
new emoji can be used, a code point must be assigned and be recognized
by the sender and receiver. However, a longer-term goal for Unicode is
that implementations should support ``embedded graphics, in addition
to the emoji characters''\autocite[Section 8, ``Longer Term
  Solutions'']{UTR51}.

In this proposal, we describe a mechanism to uniquely identify
arbitrary images within a plain-text Unicode character sequence. This
will allow implementers to create their own emoji without needing to
request and wait for the assignment of a code point. The basic idea is
to encode a globally-unique \emph{secure hash} of the emoji in a
Unicode character sequence. Once the receiver knows the image's hash,
it may already have the corresponding image, or may have a choice of
several mechanisms to retrieve it.

Our technique will gracefully degrade on legacy Unicode
implementations, but our proposal is limited to allowing Unicode to
\emph{uniquely identify} an arbitrary image. We propose to leave to
implementers the details of how to retrieve the actual image.

\section{Proposal}

\textbf{Outline}: implementations will be able to create
``implementation-defined emoji,'' and allow their users to send them
to receivers. The sender encodes a secure hash of the emoji (which
serves as an unforgeable globally unique identifier) in a sequence of
coded characters, which we call a Coded Hash of an Arbitrary Image
(CHAI). Upon receiving the CHAI sequence, the receiving implementation
may already know the corresponding image, or may have to request it
either from the sender or from a third party.

\subsection{Representing the emoji}

The first step to creating an implementation-defined emoji will be to
represent the emoji in a canonical form. This format will need to be
specified, but we do not do so here. As a straw-man, we suggest a JSON
document with three keys: \texttt{content-type} (any IANA-registered MIME
Media Type), \texttt{image} (a suggested rendering of the emoji,
interpreted as the format named in the \texttt{content-type} field), and
\texttt{name} (name of the emoji, which may be used in fallback
rendering for visually-impaired users). We call this representation
the ``emoji description.''

\subsection{Secure hash}

The implementation will identify the emoji by taking the SHA-256 hash
of the emoji description. Given a secure hash value, it is intended to be
intractable to find a different input that produces the same
hash value. This is known as ``second-preimage resistance'';
SHA-256 is believed to have strong second-preimage resistance.

\subsection{Code points allocated to express the secure hash}

We propose to allocate 65,536 ($2^{16}$) code points to represent bits
of the secure hash. Each code point will express 16 bits of the secure
hash. We envision allocating Plane 13, from U+D0000 through
U+DFFFD. (The two code points at the end of the plane are
noncharacters.) These 65,534 code points will represent the 16-bit
values $\mathrm{0000}_{16}$ through $\mathrm{FFFD}_{16}$. In addition,
we envision allocating U+EFFFC to represent the value
$\mathrm{FFFE}_{16}$, and U+EFFFD to represent the value
$\mathrm{FFFF}_{16}$. We refer to these 65,536 code points as ``CHAI
characters.'' Each code point will have a combining-character general
property. \textcolor{red}{\bf (which?)}

In UTF-8, UTF-16, and UTF-32, every code point outside the BMP
requires 32 bits to encode. As a result, the efficiency of
this scheme will be 50\% in each Unicode encoding scheme (16 bits of
the hash will consume 32 bits in memory, on disk, or on the wire).

\subsection{Encoding an implementation-defined emoji}

To code a CHAI, the sender first encodes a ``fallback'' base character
that most closely approximates the implementation-defined emoji. This
will allow rendering to degrade gracefully on receivers that do not
support CHAIs.

Next, the sender encodes the secure hash by appending between between
one and sixteen CHAI characters. This will represent a prefix of the
SHA-256 hash of the emoji description. The sender chooses how many
bits of the hash to include. (We expect 80 bits, or five CHAI
characters, to be sufficient for typical use today. The risk of
including too few bits is that the hash may no longer be globally unique,
especially in the presence of an attacker who wishes to create a different
image with the same hash prefix.)

Therefore, the actual encoding is:

\texttt { <\textit{basechar}> + <\textit{CHAI char}> + <\textit{CHAI char}> + <\textit{CHAI char}> \ldots }

where the \textit{basechar} represents a fallback base character, and
each CHAI character (from the range U+D0000 .. U+DFFFD and U+EFFFC
.. U+EFFFD) represents 16 bits of the secure hash of the emoji
description.

\subsection{Receiving and rendering a CHAI}

On receipt, the receiver determines if it already knows of an emoji
whose hash matches the prefix encoded in the CHAI characters. If so,
it displays the emoji (either using the image provided in the emoji
description, or an image known locally). Otherwise, the receiver
displays the base character while it attempts to retrieve an emoji
description whose hash matches the encoded hash prefix.

We do not specify how this retrieval will take place; we expect it to
vary based on the messaging protocol. In some protocols, it might be
easiest for the receiver to simply ask the sender to send the full
emoji description. In others, the receiver might consult an online
repository run by the implementing vendor. Alternately, the receiver
might consult a community repository where anybody can contribute any
emoji description.

Because the CHAI serves as an unforgeable globally unique identifier
for the emoji description, the receiver is intended to have confidence
that if it finds a matching emoji description from anywhere, it can
display it per the sender's intent.

Depending on the protocol, as the input arrives, the receiver may have
some ambiguity about when the sequence of CHAI characters ends.  A
receiver may choose to wait until the next non-combining character
(signaling the end of the combining character sequence), or a
protocol-defined end-of-message signal, before retrieving the emoji
description.

\subsection{Privacy}

As Unicode Technical Report \#51 notes: ``There are also privacy
aspects to implementations of embedded graphics: if the graphic itself
is not packaged with the text, but instead is just a reference to an
image on a server, then that server could track usage.''

We agree. A similar exposure could occur if the protocol provides
for the receiver to ask the sender to send an unknown emoji description;
this could unwittingly expose if anybody \emph{else} has ever sent the same
emoji to the receiver.

It is not the goal of this document to address all aspects of embedded
graphics, and we intend to leave to domain experts and implementers
the decision on how to handle these questions. For example, a client
might behave similarly to current practice in retrieving external
images in email. If the receiver already knows of the identified
emoji, it can display it, but otherwise the receiver would display the
fallback (base) character and ask the user to push a button before
retrieving the emoji description.



% Probably not needed - small enough.
%\clearpage
%\tableofcontents
%\printindex

%\clearpage
\addcontentsline{toc}{chapter}{References}
\printbibliography
\section*{Colophon}

Typeset by \LaTeX . Made with \( 100\%  \) recycled bits.
All opinions belong to the authors and do not reflect the opinions
of their associated employers.

\end{document}
