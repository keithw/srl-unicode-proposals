% !TEX TS-program = XeLaTeX
% The next lines tell TeXShop to typeset with xelatex, and to open and save the source with Unicode encoding.

%!TEX encoding = UTF-8 Unicode
% !BIB TS-program =

\documentclass[12pt]{article}

\usepackage{uproposal}


\title{Image Hash, or, the last frontier of Emoji encoding}
\author{Steven R. Loomis (individual contribution) \\
		srl@icu-project.org \url{https://srl295.github.io}}

\date{2016-04-29}                                           % Activate to display a given date or no date'




\makeindex

\begin{document}

\maketitle

\section{Introduction}
\addcontentsline{toc}{section}{Introduction}


According to the Unicode Consortium's FAQ, 

\begin{quote}
Emoji are ``picture characters'' originally associated with cellular telephone usage in Japan, but now popular worldwide.
\autocite{UnicodeFAQEmoji}
\end{quote}

The popularity of Emoji has been discussed elsewhere, but for the purposes of this document
it is worth noting that over 735 code points have been designated Emoji\autocite[Section 3, ``Which Characters are Emoji'']{UTR51} with more in the pipeline.
The stated longer-term goal for Unicode, however, is that implementations would support \textit{``embedded graphics, in addition to the emoji characters''}\autocite[Section 8, ``Longer Term Solutions'']{UTR51}.

This document aims to present a proposed implementation of said stated goal, by
a mechanism for encoding a unique embedded graphic identifier in plain text. If such a mechanism
were widely adopted,  the need for additional code point allocation for Emoji should be obviated.

\section{Non-Goals}
\addcontentsline{toc}{section}{Non-Goals}

UTR \# 51 outlines some possible use-case scenarios as well as challenges with embedded graphics \autocite[Section 8, ``Longer Term Solutions'']{UTR51}. It is not the goal of this document to
address all aspects of embedded graphics. This document will focus on those aspects related to character
encoding only, and leave to domain experts and implementers to determine standardized
approaches to topics such as privacy and security, actual data transfer of the image content,
reliability and availability, and the like.


\section{Overview}
\addcontentsline{toc}{section}{Overview}

This document proposes:

\begin{enumerate}

\item The encoding of a new base character for image transfer, 
  \texttt{ U+FFF8 EMBEDDED IMAGE BASE }
  
\item The allocation of the entire plane \texttt{ 0C } for the purpose of image hashes

\end{enumerate}

( TODO TODO )

To generate a hash, the image content (a standardized size, 128x128 png),
plus the metadata (content-type, alternates, etc) is SHA-256 hashed.

The actual encoding is:


\texttt { U+FFF8 + U+0Cxxxx + U+0Cxxxx + U+0Cxxxx ... }

%U \+ FFF8  \+  U+0C____ \+ U \+ 0C____ \+ …

where each \texttt{ U+0C } code point, from \texttt { U+0C0000 — U+0C7FFF } contains
15 bits of the SHA-256 hash.

The U+0C code points will have a combining character general category (which?).  

The more  \texttt{ U+0C }  present (up to 20 - 300 bits ) the longer and more specific the hash is.

( TODO TODO )


% Probably not needed - small enough.
%\clearpage
%\tableofcontents
%\printindex

\clearpage
\addcontentsline{toc}{chapter}{References}
\printbibliography
\section*{Colophon}

Typeset by \LaTeX . Made with \( 100\%  \) recycled bits.
All opinions belong to the authors and do not reflect the opinions
of their associated employers.

Thank you to Keith Winstein for the discussion which finally kicked off this document.



\end{document}
