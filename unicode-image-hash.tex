% !TEX TS-program = XeLaTeX
% The next lines tell TeXShop to typeset with xelatex, and to open and save the source with Unicode encoding.

%!TEX encoding = UTF-8 Unicode
% !BIB TS-program =

\documentclass[12pt]{article}

\usepackage{uproposal}


\title{Coded Hashes of Arbitrary Images\\\normalsize \sf (or: the last frontier of Emoji encoding)}
\author{
Steven R.~Loomis\\
\small srl@icu-project.org\\
\small (individual contribution)\\
\and
Keith Winstein\\
\small keithw@cs.stanford.edu\\
\small Stanford University\\
}

\date{2016-05-02}                                           % Activate to display a given date or no date'




\makeindex

\begin{document}

\maketitle

\section{Introduction}
\addcontentsline{toc}{section}{Introduction}

\begin{quote}
Emoji are pictographs (pictorial symbols) that are typically presented
in a colorful cartoon form and used inline in text. [\ldots] In
Unicode 8.0, there is a total of 1,282 emoji, which are represented
using 1,051 code points.\autocite{UTR51}
\end{quote}

Recently, there has been considerable interest in adding newly created
pictorial symbols, not found in any existing character set, to the
Unicode Standard as emoji.\footnote{\textit{E.g.,}
  \href{https://www.change.org/p/unicode-consortium-the-taco-emoji-needs-to-happen-aeb4ebc7-a323-441d-90b9-20b90c83a8c6}{Taco
    Emoji campaign}, \href{http://www.beardemoji.com/}{Beard Emoji
    campaign}, \href{http://www.dumplingemoji.org}{Dumpling Emoji
    campaign}. There are currently 79 candidate emoji that have been
  assigned tentative code points.} Advocacy groups and others request
these code points because Unicode plain text remains the dominant
interoperable interchange format for messaging. In practice, before a
new emoji can be used, a code point must be assigned and be recognized
by the sender and receiver. However, a longer-term goal for Unicode is
that implementations should support ``embedded graphics, in addition
to the emoji characters''\autocite[Section 8, ``Longer Term
  Solutions'']{UTR51}.

In this proposal, we describe a mechanism to uniquely identify
arbitrary images within a plain-text Unicode character sequence. This
will allow implementors to create their own emoji without needing to
request and wait for the assignment of a code point. The basic idea is
to encode a globally-unique \emph{secure hash} of the emoji in a
Unicode character sequence. Once the receiver knows the image's hash,
it may already have the corresponding image, or may have a choice of
several mechanisms to retrieve it.

Our technique will gracefully degrade on legacy Unicode
implementations, but our proposal is limited to allowing Unicode to
\emph{uniquely identify} an arbitrary image. We propose to leave to
implementors the details of how to retrieve the actual image.

\section{Proposal}

We propose that implementations be able to uniquely identify an
arbitrary image within a Unicode character sequence. The means will be
to encode, in a series of coded characters, a secure hash of a
canonical representation of the image.



\section{Non-Goals}
\addcontentsline{toc}{section}{Non-Goals}

UTR \# 51 outlines some possible use-case scenarios as well as challenges with embedded graphics \autocite[Section 8, ``Longer Term Solutions'']{UTR51}. It is not the goal of this document to
address all aspects of embedded graphics. This document will focus on those aspects related to character
encoding only, and leave to domain experts and implementers to determine standardized
approaches to topics such as privacy and security, actual data transfer of the image content,
reliability and availability, and the like.


\section{Overview}
\addcontentsline{toc}{section}{Overview}

This document proposes:

\begin{enumerate}

\item The encoding of a new base character for image transfer, 
  \texttt{ U+FFF8 EMBEDDED IMAGE BASE }
  
\item The allocation of the entire plane \texttt{ 0C } for the purpose of image hashes

\end{enumerate}

( TODO TODO )

To generate a hash, the image content (a standardized size, 128x128 png),
plus the metadata (content-type, alternates, etc) is SHA-256 hashed.

The actual encoding is:


\texttt { U+FFF8 + U+0Cxxxx + U+0Cxxxx + U+0Cxxxx ... }

%U \+ FFF8  \+  U+0C____ \+ U \+ 0C____ \+ …

where each \texttt{ U+0C } code point, from \texttt { U+0C0000 — U+0C7FFF } contains
15 bits of the SHA-256 hash.

The U+0C code points will have a combining character general category (which?).  

The more  \texttt{ U+0C }  present (up to 20 - 300 bits ) the longer and more specific the hash is.

( TODO TODO )


% Probably not needed - small enough.
%\clearpage
%\tableofcontents
%\printindex

\clearpage
\addcontentsline{toc}{chapter}{References}
\printbibliography
\section*{Colophon}

Typeset by \LaTeX . Made with \( 100\%  \) recycled bits.
All opinions belong to the authors and do not reflect the opinions
of their associated employers.

Thank you to Keith Winstein for the discussion which finally kicked off this document.



\end{document}
